\documentclass[11pt,singleside]{myfithesis2}
\usepackage[hyphens]{url}
\usepackage[english]{babel}  %  package  for  multilingual  support
\usepackage[utf8]{inputenc}  %  UTF-8  encoding
\usepackage[IL2]{fontenc}
\usepackage[plainpages=false,pdfpagelabels,unicode]{hyperref}
\usepackage{fancyvrb}
\usepackage{graphicx}
\usepackage{listings}
\usepackage{color}
\usepackage{textcomp}
\usepackage[small]{caption}
\usepackage{float}
\usepackage{verbatim}
 
\definecolor{listinggray}{gray}{0.9}
\definecolor{lbcolor}{rgb}{0.9,0.9,0.9}
\definecolor{darkgray}{gray}{0.3}
\definecolor{palegray}{gray}{0.85}

\hyphenation{YSoft SafeQ Sikuli Soft}

\lstset{
	numbers=left,
	backgroundcolor=\color{lbcolor},
	tabsize=4,
	rulecolor=,
	language=java,
       basicstyle=\footnotesize,
       %basicstyle=\ttfamily,
       upquote=true,
       aboveskip={\baselineskip},
       %belowskip={\baselineskip},
       columns=fixed,
       showstringspaces=false,
       extendedchars=true,
       breaklines=true,
       prebreak = \raisebox{0ex}[0ex][0ex]{\ensuremath{\hookleftarrow}},
       frame=tb,
       showtabs=false,
       showspaces=false,
       showstringspaces=false,
       identifierstyle=\ttfamily,
       keywordstyle=\color[rgb]{0,0,1},
       commentstyle=\color[rgb]{0.133,0.545,0.133},
       stringstyle=\color[rgb]{0.627,0.126,0.941},
	emph=, emphstyle=\color{black},
	captionpos=b,
}



\newcommand{\pict}[4]{
	\begin{figure}[h!]
  		\vspace{-7px}
  		\centerline{\fcolorbox{darkgray}{palegray}{\includegraphics[#3]{#2}}}
  		\caption{#1}
  		\label{#4}
	\end{figure}
}

\makeatletter
\renewcommand\paragraph{
   \vspace{-10pt}
   \@startsection{paragraph}{4}{0mm}
      {\baselineskip}
      {- 5pt}
      {\normalfont\normalsize\bfseries}
}
\makeatother
\renewcommand\lstlistingname{Source code}
\renewcommand\lstlistlistingname{List of source codes}

\hypersetup{
    colorlinks,
    citecolor=black,
    filecolor=black,
    linkcolor=black,
    urlcolor=black
}

\clubpenalty=10000
\widowpenalty=10000
\thesistitle{Making Internal Communication in a Software Company More Efficient}  %  thesis  title
\thesissubtitle{Diploma thesis}
\thesisstudent{Martin Bryndza}        %  name  of  the  author
\thesiswoman{false} %  defines  author’s  gender
\thesisfaculty{fi}
\thesisyear{2015}
\thesisadvisor{Mgr. Peter Neugebauer, MBA}  %  fill  in  advisor’s  name
\thesislang{en} %  thesis  is  in  English
\begin{document}
\FrontMatter
\ThesisTitlePage
\begin{ThesisDeclaration}
\DeclarationText
\AdvisorName
\end{ThesisDeclaration}
\begin{ThesisThanks}
I would like to thank to my supervisor Mgr. Petr Neugebauer, MBA for his valuable advices and guidance. Great thanks go to the Y Soft Corporation for giving me the opportunity to perform the unnecessary research. Many thanks also go to my colleagues from the ETNA team, who contributed to this thesis with their cooperation and willingness to try new things. These are the people who agreed to do some extra work without hesitation. I would like to send many thanks also to my colleagues from the QA team for their support in the last months of me working on this thesis. With their help I was able to have enough time for the thesis while also getting all the planned work done for each sprint. Last but not least, I would like to thank to my dear Marie for her patience with me working overnight and also to my parents for their support throughout my studies. 
\end{ThesisThanks}
\begin{ThesisAbstract}
The goal of this thesis is to streamline communication between team members and to prevent loss of productivity due to frequent interruptions in a mid-size software company, where employers are dealing with adaptation of agile approaches. The thesis should analyze internal processes as well as requirements and expectations of different stakeholders regarding productivity, especially communication gaps. Based on these identified issues, the implementation part of the work will cover design and implementation of a tool, which will enforce the team members to follow the proposed process(es). This thesis is realized in cooperation with Y Soft Corporation, a.s.
\end{ThesisAbstract}
\begin{ThesisKeyWords}
Agile, Communication, Teamwork.
\end{ThesisKeyWords}
\MainMatter
\tableofcontents %  prints  table  of  contents
%\listoffigures % prints list of pictures
%\lstlistoflistings % prints list of blocks of code
%\listoftables % prints list of tables

\chapter{Introduction}

\chapter{Effective Work}

	\section{The Cost of Interruptions}
Two - three studies on interuptions and their impact on work effectivity, stress level, workload, frustration and time pressure.

	\section{The Pomodoro Technique}
The Pomodoro Technique was created with the aim of using time as a valuable ally to acomplish what we want to do the way we want to do it, and to empower us to continually improve our work or study processes.\cite{pomodoro} The technique was invented by Francesco Cirillo in the early 80s, as a result of his own and fellow students ineffective study habits. The name `'Pomodoro'' translates as `'Tomato'' and is inspired by a kitchen timer, that usually comes in a shape of this vegetable. And it is this kitchen-timer-like device which plays an important role in the process of using the Pomodoro Technique.

The goal of the Pomodoro Technique is to provide a simple tool/process for improving productivity of not only individuals, but also whole teams. The technique is supposed to do the following\cite{pomodoro}:
\begin{itemize}
	\item alleviate anxiety linked to becoming\footnote{An abstract, dimensional aspect of time which is supported by the idea of representing time on an axis,  as we would represent spatial dimensions. This results in a concept of the duration of an event and the idea of being late (the distance of two points on the temporal axis).\cite{pomodoro}};
	\item enhance focus and concentration by cutting down on interruptions;
	\item increase awareness of your decisions;
	\item boost motivation and keep it constant;
	\item bolster the determination to achieve your goals;
	\item refine the estimation process, both in qualitative and quantitative terms;
	\item improve your work or study process
	\item strenghten your determination to keep on applying yourself in the face of complex situations
\end{itemize}

The Pomodoro Technique is founded on three basic assumptions\cite{pomodoro}:
\begin{itemize}
	\item A different way of seeing time (no longer focused on the concept of becoming) alleviates anxiety and in doing so leads to enhanced personal effectiveness. 
	\item Better use of the mind enables us to achieve greater clarity of thought, higher consciousness, and sharper focus, all the while facilitating learning. 
	\item Employing easy-to-use, unobtrusive tools reduces the complexity of applying the Technique while favoring continuity, and allows you to concentrate your efforts on the 	activities you want to accomplish. Many time management techniques fail because they subject the people who use them to a higher level of added complexity with respect to the intrinsic complexity of the task at hand.
\end{itemize}

The following objectives are required to be met one after another in order to implement the Pomodoro Technique\cite{pomodoro}.
\paragraph*{Find out how much effort an activity requires: } The traditional Pomodoro takes 25 minutes of work and a 5-minute break. First, a list of To Do Today Sheet should be created out of a pool of tasks in a Activity Inventory Sheet, which contains all the tasks ordered according to their priority. When a Pomodoro is started, it ought not to be interrupted by anybody and anything, otherwise it is considered void. The remaining time should always be clearly visible. When Pomodoro rings, it is not allowed to continue working any longer and the current activity is considered finished, at least temporarily. The successfully finished Pomodoro is marked with an X on the To Do Today Sheet. A 3-5 minute break follows, which gives time needed to ``disconnect'' from the activity, assimilate what's been learned and, for example, do something beneficial for health such as stand up and walk around. The whole cycle repeats four times, when a longer 15-30 minute break takes place. The X symbols make it possible to easily measure the total time spent on each activity, which represent the real expended effort.
\paragraph*{Cut down on interruptions: } There are two types of interruptions: internal and external. The internal interruptions are ways to procrastinate on the activity at hand, which generally is a disguised fear of not being able to finish the activity the way we want and when we want. The external interruptions are generally caused by other people. These types of interruptions share the way of dealing with them: invert the dependency on interruptions and make them depend on us. Generally, almost all of these interruptions can be delayed until the end of the currently running Pomodoro or even until later, though the interruption is considered urgent at the time it emerges. The delay isn't usually detrimental to the source of the interruption and gives an enormous advantage in terms of working more effectively. The interruption should be marked on the To Do Today Sheet and the new activity written down in order to be planned based on it's priority at the end of the Pomodoro.
\paragraph*{Estimate the effort for activities: } The long-term objective here is to successfully predict the effort that an activity requires.These estimates are based on the previously finished Pomodoros. The predicted and real number of Pomodoros spent for each activity is noted and used for better predictions in the future.
\paragraph*{Make the Pomodoro more effective and set up a timetable: } These two objectives are implemented after mastering the previous ones. They include using the first and last few minutes of each Pomodoro for reviewing the work done and setting up timetable for each day in order to prevent working late hours after wasting time in the morning.

\vspace{\baselineskip}
With Pomodoro, there is an ever-valid rule: `'Next Pomodoro will go better.'' According to the Francesco Cirillo's book The Pomodoro Technique \cite{pomodoro}, the technique has been successfully applied in various types of activities: organizing work and study habits, writing books, drafting technical reports, preparing presentations, and managing projects, meetings, events, conferences, and training courses.


\chapter{Importance of Internal Communication in a Company}

	\section{Impact of Communication on Processes}
Why communication is important at all and why it is necessary for people to be available.
	
	\section{Nonverbal Communication in Teamwork}
Why communication over chat is not enough, can lead to problems and is highly inefficient. (Gives reason why personal meetings/interruptions are important)
		
	
\chapter{Analysis of the Current State}

As an employee of Y Soft Corporation, a.s. I could define the communication patterns and customs from my point of view. However, this would certainly not be enough to propose a viable solution as there are many other aspects that are invisible for me. This chapter will give an insight on the current customs in internal communication in the company, elaborate on communication matrices and bring results of a small research.

	\section{Characteristics of Y Soft Corporation, a.s.}
In order to understand the employees, their customs and needs, first we have to look at the company itself...

		\subsection{Historical Background}
The Y Soft Corporation, a.s. is a young company, which still considers itself a startup. Many of the current employees still remember the days, when the company consisted of a few people sitting in one room mostly working on the same subject. In those times...

		\subsection{Current State}
some mixture of corporation and startup

R\&D in Brno consists of XY people. XY in QA, X in average in every team

	\section{The Nature of Internal Communication in R\&D}

		\subsection{Ad-hoc Meetings inside R\&D}
		
		\subsection{Consultations from CSS}

	\section{Unconference with Kentico Software, s.r.o.}
briefly the difference between Y Soft an Kentico in structure of R\&D + definition of their problems, ways of solving

	\section{Initial Research and Meetings}
Number of communication paths, amount of frequent and occasional communication
Define communication media (face to face, HipChat, Cisco Jabber, email, Skype)
Historical attempts (headphones, hats), time fragmentation of time of three colleagues (graphs), 

	\section{Definition of Issues to be Solved}
Deframent time, make inception of a consultation easier, prevent unnecessary communication, prevent frequent interruptions

\chapter{Proposed Solution}

	\section{The Idea}
Default state is Available, DND is available for certain period of time, users should know, when and where they can vommunicate
	
	\section{AnyOffice Tool}
Top level description of inner logic

\chapter{Purposes and Features to Reach Them}
Purposes of AnyOffice with description of features, that make them possible

\chapter{Results}
Graphs from AnyOffice containing states and consultations

\chapter{Conclusions and Future Work}

\clearpage
\phantomsection
\markright{Bibliography}
\addcontentsline{toc}{chapter}{Bibliography}
\bibliographystyle{plain}  %  sets  plain  bibliography  style
\bibliography{diplomka} %  BibTeX  database  file


\appendix

\chapter{Some chapter}



\end{document}